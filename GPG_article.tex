\documentclass[11pt]{article}
\usepackage[utf8]{inputenc}
\usepackage{amsmath}

\title{Geometrie pro počítačovou grafiku}
\author{Pavel Drábek, DRA0042}

\begin{document}
\maketitle

\section*{Průsečík přímky s koulí}
Parametrické vyjádření přímky 
\begin{align*}
  x(t) &= A_x + u_x * t\\
  y(t) &= A_y + u_y * t\\
  z(t) &= A_z + u_z * t
\end{align*}
Parametrické vyjádření koule 
\begin{align*}
  (x-s_x)^2 + (y-s_y)^2 + (z-s_z)^2 &= r^2
\end{align*}
Dosadíme rovnici přímky do rovnici koule 
\begin{align*}
  (A_x + u_x * t-s_x)^2 + (A_y + u_y * t-s_y)^2 + (A_z + u_z * t-s_z)^2 = r^2
\end{align*}
Vyjádříme parametr $t$ pomocí kvadratické rovnice
\begin{align*}  
  at^2 + bt + c &= 0\\
  t_{1,2} &= \frac{-b\pm\sqrt{D}}{2ac}\\
  D &= b^2-4ac
\end{align*}
\begin{align*}  
  a &= t^2(u_x^2 + u_y^2 + u_z^2)\\
  b &= 2t(u_x(A_x-s_x) + u_y(A_y-s_y) + u_z(A_z-s_z))\\
  c &= (A_x - s_x)^2 + (A_y - s_y)^2 + (A_z - s_z)^2 - r^2\\  
\end{align*}
Pokud $D >= 0$ přímka protína kouli. Parametr $t$ dosadíme do parametrické vyjádření přímky a získáme průsečík, kde přímka protíná kouli. Pokud budeme uvažovat nad vektorem $u$ jako normalizovaným směrovým vektorem přímky a budeme chtít nalézt průsečík pouze ve směru $u$ od bodu $A$, parametr $t \in <0,\infty>$. 
\end{document}