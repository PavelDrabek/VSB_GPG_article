\documentclass[11pt]{article}
\usepackage[utf8]{inputenc}
\usepackage{amsmath}

\title{Geometrie pro počítačovou grafiku}
\author{Pavel Drábek, DRA0042}

\begin{document}
\maketitle
\newpage

\tableofcontents
\newpage

\section{Průsečík přímky s kulovou plochou}
Cílem této kapitoly je nahradit metodu rtcIntersect z knihovny Embree pro výpočet průsečíku přímky s kulovou plochou zadanou analyticky. 

\subsection{Parametrické vyjádření přímky} 
Parametrické  vyjádření přímky je dáno bodem $A$, vektorem $\vec{u}$, parametrem $t$ a předpisem: 
\begin{equation} \label{eq:primka}
  X(t) = A + \vec{u} * t  \qquad t \in \langle0;1\rangle
\end{equation}
Kdy dosazením parametru $t$ do rovnice získáme bod $X$, který je od bodu $A$ vzdálený $t |\vec{u}|$ ve směru vektoru $\vec{u}$.

\subsection{Parametrické vyjádření kulové plochy }
Parametrické vyjádření kulové plochy je dáno středem $S=[S_x, S_y, S_z]$ kulové plochy, poloměrem kulové plochy $r$ a předpisem: 
\begin{equation} \label{eq:kulovaplocha}
  (x-S_x)^2 + (y-S_y)^2 + (z-S_z)^2 = r^2
\end{equation}
Složky $x$, $y$, $z$ tvoří souřadnice bodu, o kterém chceme zjistit zda leží na kulové ploše.

\subsection{Výpočet průsečíku}
Pokud chceme spočítat průsečík přímky a kulové plochy, musíme rovnice přímky \eqref{eq:primka} a kulové plochy \eqref{eq:kulovaplocha} sloučit. Rovnice pracuje s jednotlivými složkami bodů, takže si rozepíšeme rovnici přímky na jednotlivé složky: 
\begin{align*}
  x(t) &= A_x + u_x * t\\
  y(t) &= A_y + u_y * t\\
  z(t) &= A_z + u_z * t
\end{align*}
Nyní můžeme dosadit rovnici přímky \eqref{eq:primka} do rovnice kulové plochy \eqref{eq:kulovaplocha}: 
\begin{equation}\label{eq:primkakulovaplocha} 
  (A_x + \vec{u}_x * t-s_x)^2 + (A_y + \vec{u}_y * t-s_y)^2 + (A_z + \vec{u}_z * t-s_z)^2 = r^2
\end{equation}
Abychom získali hodnoty parametru $t$, převedeme rovnici do tvaru kvadratické rovnice: 
\begin{align}  
  t_{1,2} &= \frac{-b\pm\sqrt{D}}{2ac} \\
  D &= b^2-4ac
\end{align}
Získané koeficienty $a$, $b$, $c$ z upravené rovnice \eqref{eq:primkakulovaplocha} můžeme dosadit do kvadratické rovnice pro získání parametru $t$.  
\begin{align*}  
  a &= (u_x^2 + u_y^2 + u_z^2)\\
  b &= 2(u_x(A_x-s_x) + u_y(A_y-s_y) + u_z(A_z-s_z))\\
  c &= (A_x - s_x)^2 + (A_y - s_y)^2 + (A_z - s_z)^2 - r^2\\  
\end{align*}
Spočítaný diskriminant $D$ z rovnice \eqref{eq:primkakulovaplocha} může nabývat tří hodnot: 
\begin{enumerate}
\item $D < 0$: přímka je mimoběžná (neexistuje průsečík)
\item $D = 0$: přímka je tečna (existuje právě jeden průsečík)
\item $D > 0$: přímka je sečna (existují 2 průsečíky)
\end{enumerate}
Nyní můžeme spočítat parametr $t$. Pokud chceme spočítat průsečík ve směru přímky, musí být parametr $t\ge0$. V opačném případě přímka v daném směru kulovou plochu neprotíná.
Pokud vyjdou 2 parametry $t_1$, $t_2$ a chceme bod, který protne přímka v daném směru první, pak platí: 
\begin{equation*}
t = MIN(t_1, t_2) \qquad t_1,t_2 \ge 0
\end{equation*}
Kde funkce $MIN$ vrací menší hodnotu ze zadaných parametrů. 

\section{Barycentrické souřadnice}
Barycentrické souřadnice jsou váhové koeficienty vyjadřující blízkost vnitřního bodu $X$ trojúhelníku $ABC$ k jeho jednotlivých vrcholům $A$, $B$, $C$.
\subsection{Parametrické vyjádření}
Barycentrické souřadnice definujeme předpisem: 
\begin{equation*}
X = \alpha A + \beta B + \gamma C
\end{equation*}
\begin{equation*}
\alpha + \beta + \gamma = 1  \qquad  \alpha, \beta, \gamma \in \langle0;1\rangle 
\end{equation*}
\subsection{Odvození souřadnic}
Pokud známe všechny body trojúhelníku, můžeme odvodit koeficienty $\alpha$, $\beta$, $\gamma$. Nejprve upravíme jednu rovnici tak, abychom ji mohli odečíst od druhé: 
\begin{align*}
X_x = \alpha A_x + \beta B_x + (1 - \alpha - \beta) C_x \\
X_y = \alpha A_y + \beta B_y + (1 - \alpha - \beta) C_y 
\end{align*}
Následně rešíme soustavou dvou rovnic o dvou neznámých: 
\begin{align*}
\alpha &= \frac{(B_y-C_y)(X_x-C_x) + (C_x-B_x)(X_y-C_y) }{(B_y-C_y)(A_x-C_x) + (C_x-B_x)(A_y-C_y)}\\
\beta &= \frac{(C_y-A_y)(X_x-C_x) + (A_x-C_x)(X_y-C_y) }{(B_y-C_y)(A_x-C_x) + (C_x-B_x)(A_y-C_y)}\\
\gamma &= 1 - \alpha - \beta
\end{align*}
\end{document}