\section{Barycentrické souřadnice}
Barycentrické souřadnice jsou váhové koeficienty vyjadřující blízkost bodu $X$ trojúhelníku $ABC$ k jeho jednotlivých vrcholům $A$, $B$, $C$.
\subsection{Analytické vyjádření}
Barycentrické souřadnice definujeme předpisem: 
\begin{equation*}
X = \alpha A + \beta B + \gamma C,
\end{equation*}
\begin{equation*}
\alpha + \beta + \gamma = 1,  \qquad  \alpha, \beta, \gamma \in \langle0;1\rangle. 
\end{equation*}
\subsection{Odvození souřadnic}
Pokud známe body $A$, $B$, $C$ trojúhelníku, můžeme odvodit koeficienty $\alpha$, $\beta$, $\gamma$. Nejprve upravíme jednu rovnici tak, abychom ji mohli odečíst od druhé: 
\begin{align*}
X_x = \alpha A_x + \beta B_x + (1 - \alpha - \beta) C_x, \\
X_y = \alpha A_y + \beta B_y + (1 - \alpha - \beta) C_y.
\end{align*}
Následně rešíme soustavou dvou rovnic o dvou neznámých: 
\begin{align*}
\alpha &= \frac{(B_y-C_y)(X_x-C_x) + (C_x-B_x)(X_y-C_y) }{(B_y-C_y)(A_x-C_x) + (C_x-B_x)(A_y-C_y)},\\
\beta &= \frac{(C_y-A_y)(X_x-C_x) + (A_x-C_x)(X_y-C_y) }{(B_y-C_y)(A_x-C_x) + (C_x-B_x)(A_y-C_y)},\\
\gamma &= 1 - \alpha - \beta.
\end{align*}